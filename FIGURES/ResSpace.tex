% \begin{landscape}
% \begin{figure}[htb]
% 	%\caption{Ergebnisse erste Iteration, spezifische W\"armebedarf. Berechnet 
% 	% mit Geb\"audetypologien}
% 	\centering
% 	\includegraphics[height=0.9\textheight]{FIGURES/Btyp/HH}
% 	\caption{Detail from results from the first iteration showing specific heat
% 		demand using buildings typologies}
% \end{figure}
% \end{landscape}
% 
% \begin{landscape}
% \begin{figure}[htb]
% 	% \caption{Ergebnisse erste Iteration, spezifische W\"armebedarf. Berechnet 
% 	% mit Geb\"audetypologien}
% 	\centering
% 	\includegraphics[height=0.9\textheight]{FIGURES/Btyp/Kirchdorf_b}
% 	\caption{Detail of results from the first simulation iteration, showing heat
% 		density for selected urban area using building typologies}
% \end{figure}
% \end{landscape}

\begin{figure}
\begin{tabular}{p{0.5\linewidth}p{0.5\linewidth}} 
\def\svgwidth{\linewidth}
	\import{FIGURES/}{MapHeatDemandEco.pdf_tex}&
\def\svgwidth{\linewidth}
	\import{FIGURES/}{MapGasConsumption.pdf_tex}\\
\end{tabular}
\begin{footnotesize}
\textbf{(a)} Map showing the estimated heat demand in $MWh$ using typology
EcoFYS. 
The red circles mark:
(1) an over estimation of heat demand for non residential buildings; and
(2) identification of a (residential) heat spot in the urban area. 
\textbf{(b)} Map showing the buildings connected to the gas grid (pink)\\
\end{footnotesize}
% 	\label{fig:map-Blesl}
\caption{Maps showing the estimated heat demand and the monitored gas
consumption}
\label{fig:mapsConDem}
\end{figure}
