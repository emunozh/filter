% !TEX TS-program = pdflatex
% !TEX encoding = UTF-8 Unicode

%%\documentclass[preprint,review,12pt,authoryear]{elsarticle}

%% Use the option review to obtain double line spacing
\documentclass[authoryear,preprint,review,12pt]{elsarticle}

%% The amssymb package provides various useful mathematical symbols
%\usepackage{amssymb}
\usepackage{amsmath, amsfonts, amsthm, amssymb} 
\usepackage{lineno}
\usepackage{url}
%\usepackage{array}
%\usepackage{mathptmx}

%% Additional packages
% tables
\usepackage{booktabs} % for much better looking tables
\usepackage{longtable}
\usepackage{multicol} % multiple columns in tables
\usepackage{multirow} % multiple rows in tables
\usepackage{rotating} % sideways tables
\usepackage{framed, color}
\usepackage[dvipsnames,table]{xcolor}
% Rotate elements in a table
\usepackage{adjustbox}
\newcolumntype{R}[2]{%
    >{\adjustbox{angle=#1,lap=\width-(#2)}\bgroup}%
    l%
    <{\egroup}}
\newcommand*\rot{\multicolumn{1}{R{90}{1em}}}
% figures
\usepackage{import} % import figures created in inskape
\usepackage{tikz} % draw figures exported from R
\usepackage[percent]{overpic} % needed for graph composition
%% bib
%\usepackage[nosectionbib, tocbib, unnumberedbib]{apacite} %% APA citation style

%\journal{Energy and Buildings}
\journal{}

\begin{document}

\begin{frontmatter}

%% Title, authors and addresses
\title{Allocating heat consumption of residential buildings in space with help
    of a filter array to determinate the building type in a digital cadaster}

\author[HCU]{M. Esteban Mu\~{n}oz H.\corref{cor1}}
\ead{marcelo.hidalgo@hcu-hamburg.de}
\author[HCU]{Ivan Dochev}
\ead{ivan.dochev@hcu-hamburg.de}
\author[HCU]{Irene Peters}
\ead{irene.peters@hcu-hamburg.de}
\cortext[cor1]{Corresponding author}

\address[HCU]{HafenCity Universit\"{a}t, Hamburg, DE}
  %Technical Urban Infrastructure Systems Group,

\begin{abstract}
In this paper we present a method to allocate heat demand of residential
buildings in space. The buildings and their characteristics are available from
the city digital cadaster.\\

Based on the building characteristics an appropriate \textit{building type} is
selected. In Germany the use of building typologies for the estimation of heat
demand is a common practice. Nonetheless, this practice requires the user to
make a subjective decision regarding the choice for a type of building. In
this paper we present a method to automatically determinate the building type
given some characteristics of the buildings and a pre-defined typology. This method
allows us to: (1) systematically compare results of different building
typologies; (2) process large amount of building data from a digital cadaster
without binding the method to the cadaster data structure, making the method
transferable to other cities with a digital cadaster; and (3) because the
nature of our method is stochastic, we are able to represent not only the best
estimate of heat demand but the corresponding uncertainty level of the area and
of the typology.  \\

We expect to expand this method for the use of urban typologies able to
estimate and allocate heat demand of urban morphologies without the need of a
digital cadaster, expanding so the accessibility to this method to a global
scale.\\

The presented results in this paper have shown that: (1) national typologies
under estimate heat demand of the single buildings in the selected urban area,
while typologies developed for specific regions over estimate heat demand of
the same urban area; (2) national typologies perform better at an aggregated
level, while regional typologies perform better at a micro level (3) an
identification of ``hot spots'' in small urban areas is possible.  (4) for a
comprehensive identification of retrofit areas in urban spaces we need to
integrate: (a) the tertiary sector and (b) possible heat sources that can
supply the demand of these ``hot spots''.
\end{abstract}

\begin{keyword}
    Digital Cadaste \sep Residential Heat Demand \sep Building Typologies \sep
    Building Stock
\end{keyword}

\end{frontmatter}
\begin{linenumbers}
    \input{body.tex}
    \section{Extending the Method to Classify the Building Stock}

\subsection{The Need to Include Non Residential Building in the Analysis Scope}\label{sub-section:NR}

Many of the developed typologies have extended their approach to the tertiary
sector~\cite{Loga.2011, Hermelink.2011}, as it is this sector that will be
mixed in urban areas with the residential sector. An estimation of the
industrial sector will differ too much from the typology approach in order to be
include into this methodology. A different approach is presented
by~\cite{Blesl.2007}, in which the authors attempt to classify the tertiary
sector by the required temperature of the processes within the building.  This
approach is very interesting, and an automatic classification of buildings by
the driving needed temperature may be possible thanks to the comprehensive
classification of building use in the digital cadaster (231 classes).  A
parallel comprehensive classification of building uses and their corresponding
energy demand was developed by~\cite{Zeine.2007}.  The connection of these both
data sets may be an interesting option for the estimation of non residential
buildings.\\

Another interesting alternative, especially for an urban setting without a
comprehensive digital cadastre, can be the use of urban typologies.
% A tiny contribution by Ivan Dochev
An example for such a typology is the one developed by~\cite{Hegger.2014}, which
divides the urban territory into areas with specific heat demand and potential
for renewable energy sources. The heat demand is based upon the character of
the urban fabric --for example-- predominantly single family houses, terrace
houses, prefabricated blocks etc. The potential for renewables is derived again
from the character of the areas with the amount of non built up areas being a
signal for the potential of geothermal and biomass energy and the typical roof
form (slope, aspect), for the given area, --a signal for solar energy
potential. In this typology, however, the heat demand of the non residential
buildings is taken down to the building level, being analyzed as ``single-elements''
rather than urban areas (with one urban area exception, in which the demand of
tertiary sector is calculated per area of the urban type). Since almost the
entire heat demand of these urban areas does come from the buildings (and the
non-residential sector is actually tackled at the building level), it becomes
arguable if the switching from buildings to areas presents a benefit to the
analysis. It can be argued that it is useful for areas without a digital
cadastre, however the lack of such a cadastre may render the assignment of the
urban areas into types impossible or at least with low precision of results.
Therefore, simply methodologically, observing the heat demand for urban areas
rather than buildings may not bring benefits precisely because areas do not
demand heat, apart from the buildings in them (there are marginal exceptions,
street lightning demands electrical power for example). Looking at the supply
of heat however, the urban area typology approach presents a more solid case.
Generally, the supply of a building is external to it, thus exploring the areas
outside of the buildings as well, makes more sense. Therefore,
%
if the aim of the analysis relies not on the single buildings but on entire
urban areas taking into account not only the single heat demand of the
buildings but the underlying infrastructure supporting the area, a more
holistic approach, using areas rather than buildings, may be useful. \citeA{Roth.1980}
developed an urban typology design for the dimension of
district heating systems. In Germany and Switzerland, the use of urban
typologies for the description of the building stock and the underlying
infrastructure has a long tradition.
For the use of
urban typologies in relationship with: (1) heat supply, see~\cite{Roth.1980,
Sieverts.1980, Blesl.2002, Jentsch.2008}; (2) for the estimation of the
potential of solar power generation, see~\cite{Everding.2004}; (3) a general
description of the potential for renewable energies in open spaces,
see~\cite{Genske.2009}; (4) transportation volume estimation and transportation
patterns, see~\cite{Marconi.2006, Krug.2006}; and (5) for a description of
general infrastructure networks see~\cite{Buchert.2004, Ecoplan.2000,
Einig.2006, Erhorn.2011}.\\

\subsection{Next Steps --- Expanding the Method}\label{sub-section:next}

In this sub section we discuss the use of building typologies at a European
and global level. We argue the need to work on method rather than on specific
national building typologies, making a focus on available data of the single
countries.  We also make a small excursion to the use of urban typologies for
the estimation of heat demand and highlight the benefits and problems of using
urban typologies rather that building typologies.\\

Having access to a detailed digital cadaster of the city offers a vast set of
possibilities for the simulation of heat demand of urban areas.  A simple
balance of heat demand can be performed without much effort and with a relative
low input of data.  Much of the needed data for an energy balance of a single
building can be recover through the digital cadaster.  The missing data for
such a computation are the U-values of the single building components.  These
building typologies are not constructed for this purpose.  There is an
alternative to the use of building typologies, this is the use of a building
component typology.  For Germany there is a regional material catalog developed
by the Center for Environmental Friendly Construction\footnote{(Zentrum für
    Umweltbewusstes Bauen e.V.)
    \url{http://www.zub-kassel.de/}}~\cite{Klauss.2009}.
With help of this catalog we could simulate heat consumption upon building
components rather than upon building typologies.  First attempts have been
performed by our research team~\cite{MunozH.2015.MEQ}.
The simulation at such level of detail may foster further insights on different
topics needed for a holistic understanding of urban systems.  This method could
be applied for the simulation of material flows in urban systems as well as an
estimation of retrofit cost and the construction and demolition waste arising
from such retrofits.\\
%TODO: support with some literature about CDW

%TODO: the use of building typologies for datarecovery
%\cite{Meinel.2008}
% Automatische Ableitung von stadtstrukturellen Grundlagendaten und Integration
% in einem geographischen Informationssystem
% \cite{YiminChen.2011}
% Estimating the relationship between urban forms and energy consumption: A case
% study in the Pearl River Delta, 2005–2008

%%
\section*{Bibliography}
\bibliographystyle{elsarticle-harv} 
\bibliography{filter.bib}

\section*{Annex}
\begin{longtable}{ll | lllllll | lllll | l}

  \centering
  \caption[Building typology - EcoFYZ Parameters (1/2)]{
  Building typology for the city of Hamburg developed by EcoFYZ showing the 
  parameters of each Typ, in percentage.
  \cite[pp.~18]{Hermelink.2011}}
  \label{tab:EcoFYZParam1}%
    \begin{tabular}{ll | lllllll | lllll | l}
    \cmidrule{1-14}
    & Year (BAJ) & \multicolumn{7}{c|}{Floor (AOG)} & \multicolumn{5}{c|}{Roof (DAF)}\\
    &            & \begin{sideways}01--02 \end{sideways}&
    			   \begin{sideways}03--03 \end{sideways}& 
    			   \begin{sideways}04--05 \end{sideways}& 
    			   \begin{sideways}06--09 \end{sideways}&
    			   \begin{sideways}10--13 \end{sideways}&
    			   \begin{sideways}14--15 \end{sideways}&
    			   \begin{sideways}16--20 \end{sideways}& 
					sa & m & w & f & so & %\\
    \multirow{45}{*}{
	\begin{sideways}
	    \begin{minipage}{19cm}
			\begin{footnotesize}
				$^*$ MFH-Hochhaus, massiv.
				(sa) pitched roof, ``Satteldach'';
				(m) curg roof, ``Mansardendach'';
				(w) hip roof, ``Walmdach'';
				(f) flat roof, ``Flachdach'';
				(so) other, ``Sonstiges''.
			\end{footnotesize}
		\end{minipage}
	\end{sideways}
}\\
\cmidrule{1-14}
%\midrule
\multirow{8}{*}{\begin{sideways} Freist. EFH / DHH \end{sideways}}
\multirow{8}{*}{\begin{sideways} Semi detached / \end{sideways}}
\multirow{8}{*}{\begin{sideways} Single family house \end{sideways}}
&\textless~1918     & 94 & 6 & 0 & 0 & 0 & 0 & 0 & 51 & 13 & 26 & 5 & 5 \\ 
&1919--1948  & 99 & 2 & 0 & 0 & 0 & 0 & 0 & 60 & 5 & 26 & 4 & 5 \\ 
&1949--1957  & 99 & 1 & 0 & 0 & 0 & 0 & 0 & 71 & 2 & 19 & 4 & 4 \\ 
&1958--1968  & 98 & 2 & 0 & 0 & 0 & 0 & 0 & 65 & 3 & 20 & 9 & 3 \\ 
&1969--1978  & 99 & 2 & 0 & 0 & 0 & 0 & 0 & 61 & 3 & 24 & 10 & 3 \\ 
&1979--1983  & 99 & 1 & 0 & 0 & 0 & 0 & 0 & 65 & 4 & 22 & 6 & 3 \\ 
&1984--1994  & 99 & 1 & 0 & 0 & 0 & 0 & 0 & 47 & 5 & 42 & 3 & 2 \\ 
&\textgreater~1995     & 98 & 2 & 0 & 0 & 0 & 0 & 0 & 50 & 3 & 36 & 5 & 6 \\ 
\cmidrule{1-14}
\multirow{8}{*}{\begin{sideways} Reihenhaus \end{sideways}}
\multirow{8}{*}{\begin{sideways} Terrace house \end{sideways}}
&\textless~1918     & 84 & 13 & 3 & 0 & 0 & 0 & 0 & 37 & 24 & 14 & 23 & 2 \\ 
&1919--1948  & 95 & 5 & 0 & 0 & 0 & 0 & 0 & 68 & 3 & 19 & 10 & 1 \\ 
&1949--1957  & 99 & 1 & 0 & 0 & 0 & 0 & 0 & 85 & 1 & 2 & 11 & 1 \\ 
&1958--1968  & 100 & 0 & 0 & 0 & 0 & 0 & 0 & 77 & 1 & 1 & 20 & 1 \\ 
&1969--1978  & 97 & 3 & 0 & 0 & 0 & 0 & 0 & 52 & 2 & 1 & 43 & 2 \\ 
&1979--1983  & 96 & 4 & 0 & 0 & 0 & 0 & 0 & 79 & 7 & 1 & 10 & 2 \\ 
&1984--1994  & 98 & 2 & 0 & 0 & 0 & 0 & 0 & 72 & 12 & 9 & 2 & 5 \\ 
&\textgreater~1995     & 76 & 23 & 0 & 0 & 0 & 0 & 0 & 50 & 0 & 6 & 24 & 19 \\ 
\cmidrule{1-14}
\multirow{8}{*}{\begin{sideways} MFH-Einzelhaus \end{sideways}}
\multirow{8}{*}{\begin{sideways} Single family house \end{sideways}}
&\textless~1918     & 49 & 38 & 11 & 1 & 0 & 0 & 0 & 30 & 29 & 25 & 10 & 6 \\ 
&1919--1948  & 69 & 26 & 4 & 0 & 0 & 0 & 0 & 24 & 12 & 53 & 7 & 4 \\ 
&1949--1957  & 68 & 22 & 10 & 1 & 0 & 0 & 0 & 44 & 8 & 28 & 17 & 3 \\ 
&1958--1968  & 68 & 21 & 8 & 3 & 0 & 0 & 0 & 57 & 8 & 13 & 19 & 3 \\ 
&1969--1978  & 59 & 32 & 8 & 0 & 0 & 0 & 0 & 44 & 8 & 12 & 34 & 2 \\ 
&1979--1983  & 65 & 26 & 8 & 1 & 0 & 0 & 0 & 39 & 24 & 15 & 20 & 2 \\ 
&1984--1994  & 72 & 21 & 6 & 1 & 0 & 0 & 0 & 36 & 27 & 24 & 7 & 6 \\ 
&\textgreater~1995     & 48 & 31 & 18 & 2 & 0 & 0 & 0 & 31 & 6 & 22 & 29 & 12 \\ 
\cmidrule{1-14}
\multirow{8}{*}{\begin{sideways} MFH-Wohnblock \end{sideways}}
\multirow{8}{*}{\begin{sideways} Block family house \end{sideways}}
&\textless~1918     & 6 & 19 & 67 & 8 & 0 & 0 & 0 & 19 & 66 & 2 & 11 & 2 \\ 
&1919--1948  & 13 & 26 & 51 & 9 & 0 & 0 & 0 & 29 & 40 & 11 & 18 & 3 \\ 
&1949--1957  & 7 & 25 & 63 & 5 & 0 & 0 & 0 & 51 & 20 & 11 & 17 & 1 \\ 
&1958--1968  & 10 & 25 & 59 & 7 & 0 & 0 & 0 & 48 & 28 & 5 & 18 & 2 \\ 
&1969--1978  & 4 & 22 & 65 & 8 & 0 & 0 & 0 & 34 & 39 & 6 & 19 & 2 \\ 
&1979--1983  & 6 & 23 & 63 & 9 & 0 & 0 & 0 & 35 & 38 & 7 & 17 & 3 \\ 
&1984--1994  & 7 & 23 & 61 & 8 & 0 & 0 & 0 & 34 & 43 & 5 & 14 & 4 \\ 
&\textgreater~1995     & 25 & 31 & 34 & 9 & 0 & 0 & 0 & 30 & 18 & 6 & 38 & 8 \\ 
\cmidrule{1-14}
\multirow{8}{*}{\begin{sideways} MFH-Gruppenhaus \end{sideways}}
\multirow{8}{*}{\begin{sideways} Group family house \end{sideways}}
&\textless~1918     & 6 & 33 & 55 & 6 & 0 & 0 & 0 & 41 & 38 & 10 & 7 & 4 \\ 
&1919--1948  & 8 & 28 & 54 & 10 & 0 & 0 & 0 & 33 & 4 & 37 & 25 & 1 \\ 
&1949--1957  & 12 & 23 & 62 & 3 & 0 & 0 & 0 & 62 & 1 & 20 & 17 & 1 \\ 
&1958--1968  & 17 & 43 & 38 & 3 & 0 & 0 & 0 & 59 & 2 & 4 & 34 & 1 \\ 
&1969--1978  & 12 & 29 & 49 & 9 & 0 & 0 & 0 & 35 & 3 & 11 & 51 & 1 \\ 
&1979--1983  & 12 & 30 & 48 & 10 & 0 & 0 & 0 & 44 & 11 & 11 & 33 & 1 \\ 
&1984--1994  & 19 & 37 & 40 & 5 & 0 & 0 & 0 & 49 & 18 & 15 & 15 & 3 \\ 
&\textgreater~1995     & 14 & 33 & 45 & 9 & 0 & 0 & 0 & 29 & 4 & 13 & 45 & 9 \\ 
\cmidrule{1-14} 
\multirow{2}{*}{\begin{sideways} HH$^*$ \end{sideways}}
&1958--1968  & 0 & 0 & 0 & 55 & 32 & 5 & 6 & 5 & 3 & 0 & 89 & 3 \\ 
&1969--1978  & 0 & 0 & 1 & 39 & 47 & 7 & 6 & 0 & 0 & 0 & 99 & 0 \\
\cmidrule{1-14}
    \end{tabular} 
\end{table}%

\end{linenumbers}
\end{document}

\endinput
